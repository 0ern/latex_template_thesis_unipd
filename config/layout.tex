\newfont{\ttvar}{cmvtt10 scaled 1200}   % nuovo carattere tipi courier a spaziatura variabile per le dimostrazioni

% margini senato
\textwidth       =  14.50 cm            % larghezza 21 cm - 4 cm (sinistro) - 2.5 (destro)
\textheight      =  23.10 cm            % altezza 29.7 cm - 3 cm (superiore) - 2 (inferiore)
\topmargin       =   0.00 cm            % margine superiore 3 cm diminuito di 1 inch
\oddsidemargin   =   1.46 cm            % margine sinistro 4 cm diminuito di 1 inch
\evensidemargin  =  -0.04 cm            % margine destro 2.5 cm diminuito di un inch

\setlength{\headsep}{1.0cm}
\setlength{\footskip}{1.0cm}
\setlength{\headheight}{15pt} % ...at least 14.9999pt
\parindent = 0.7cm
\captionmargin = 0.7cm

%Interlinea 1.5
\onehalfspacing

% stile pagina
\pagestyle{fancy}
\renewcommand{\chaptermark}[1]{\markboth{\chaptername\ \thechapter.\ #1 }{}}
\renewcommand{\sectionmark}[1]{\markright{\thesection\ #1}{}}
\fancyhead{}
\fancyhead[LE,RO]{\sffamily \thepage}
\fancyhead[RE]{\sffamily \leftmark}
\fancyhead[LO]{\sffamily \rightmark}
\fancyfoot{}%[CE,CO]{\sffamily \thepage}

%\renewcommand{\headrulewidth}{0pt}
% ridefinisco lo stile plain
\fancypagestyle{plain}{ \fancyhead{} \fancyfoot{}
                        \fancyfoot[C]{\sffamily \thepage}
                        \renewcommand{\headrulewidth}{0pt}
                      }

% stile per i titoli
%\allsectionsfont{\sffamily \raggedright}

%definisco stile listati di codice
\lstdefinestyle{mystyle}{
                        backgroundcolor=\color[RGB]{240,240,240}, %Red, Green, Blue from 0 to 255
                        basicstyle=\color{black}\ttfamily\footnotesize,
                        commentstyle=\color{OliveGreen}\textit\tiny,
                        keywordstyle=\color{blue},
                        identifierstyle=\color{black},
                        stringstyle=\color{violet},
                        breakatwhitespace=false,         
                        breaklines=true,
                        captionpos=b, %bottom, top
                        rulecolor=\color{black},
                        frame=trBL, %Top,Right,Bottom,Left (Case sensitive)
                        frameround=tttt, %t=round corner, f=straight corner
                        keepspaces=true,                 
                        numbers=left,                    
                        numbersep=10pt,
                        numberstyle=\color{gray},
                        showspaces=false,                
                        showstringspaces=false,
                        showtabs=false,                  
                        tabsize=1,
                        columns=fullflexible %makes code copy and pastable
                        }

\lstset{style=mystyle}
\lstset{emph={RandomForestClassifier, RandomizedSearchCV, GridSearchCV},emphstyle=\underbar}

\renewcommand{\lstlistingname}{Codice} % Per rinominare la descrizione dei codici
\renewcommand{\lstlistlistingname}{Elenco dei codici} % Per rinominare la descrizione dei codici