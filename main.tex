\documentclass[a4paper, 12pt, hyphens]{book}

% Pacchetti utili
% Codifica
\usepackage[utf8]{inputenc}

% Lingua, sostituire italian con english nel caso in cui la tesi sia scritta in inglese
\usepackage[italian]{babel}
\usepackage[T1]{fontenc}

% Stile font
\usepackage{lmodern}

% Colori
\usepackage[usenames]{color}
\usepackage[dvipsnames]{xcolor}
\usepackage{colortbl}

% Pacchetto per definire layout di pagina
%\usepackage{sectsty}
\usepackage{fancyhdr}
\usepackage[left=3cm, right=3cm, bottom=3cm]{geometry}

% Spazia linee all'interno del documento
\usepackage{setspace}

% Impostazioni note a piè pagina
\usepackage[stable, multiple]{footmisc}

% Crea link ipertestuali
\usepackage[hidelinks, hyperfootnotes=false]{hyperref}

% PDF in formato PDF/A-1b
%\usepackage[a-1b]{pdfx}

% Inclusione immagini
\usepackage{graphicx}

% Didascalie immagini
\usepackage[hang, small, sf, labelfont=bf]{caption}
\usepackage{subcaption}

% Formattazione url
\usepackage{url}

% Inserimento formule
\usepackage{amsmath}
\usepackage{mathrsfs}

% Listati di codice
\usepackage{verbatim}
\usepackage{listings}

%Inserimento pseudocodice
\usepackage{algorithm}
\usepackage{algpseudocode}

% Citazione frasi
\usepackage{csquotes}

% Citazioni e riferimenti a label
%\usepackage{cite} % Don't use with biblatex package
%\usepackage[english]{varioref}

% Bibliografia
% This ----------------------------------------------------------
\usepackage[backend=biber, style=numeric, sorting=none, backref=true]{biblatex} % biber or bibtex - numeric or alphabetic - none or nty
\addbibresource{chapters/references.bib}
% Or this -------------------------------------------------------
% \usepackage{natbib} 
% \usepackage[nottoc]{tocbibind}
% \bibliographystyle{plain}

\usepackage{chngcntr}
% \counterwithin{figure}{section} % Change figure numbering by section (with chngcntr pkg)
% \counterwithin{table}{section} % Change table numbering by section (with chngcntr pkg)
% \numberwithin{equation}{section} % Change equation numbering by section (with amsmath pkg)
\counterwithout{footnote}{chapter} % Change footnote numbering by chapter (with chngcntr pkg)

% Inserimento di Lorem ipsum nel testo
\usepackage{lipsum}

% Prima pagina bianca
\usepackage{afterpage}
\newcommand\myemptypage{
                        \null
                        \thispagestyle{empty}
                        \addtocounter{page}{-1}
                        \newpage
                        }

\usepackage{tabularx} 

%\usepackage{marginnote}

% Impostazioni dei margini, definizione di colori e stili
\newfont{\ttvar}{cmvtt10 scaled 1200}   % nuovo carattere tipi courier a spaziatura variabile per le dimostrazioni

% margini senato
\textwidth       =  14.50 cm            % larghezza 21 cm - 4 cm (sinistro) - 2.5 (destro)
\textheight      =  23.10 cm            % altezza 29.7 cm - 3 cm (superiore) - 2 (inferiore)
\topmargin       =   0.00 cm            % margine superiore 3 cm diminuito di 1 inch
\oddsidemargin   =   1.46 cm            % margine sinistro 4 cm diminuito di 1 inch
\evensidemargin  =  -0.04 cm            % margine destro 2.5 cm diminuito di un inch

\setlength{\headsep}{1.0cm}
\setlength{\footskip}{1.0cm}
\setlength{\headheight}{15pt} % ...at least 14.9999pt
\parindent = 0.7cm
\captionmargin = 0.7cm

%Interlinea 1.5
\onehalfspacing

% stile pagina
\pagestyle{fancy}
\renewcommand{\chaptermark}[1]{\markboth{\chaptername\ \thechapter.\ #1 }{}}
\renewcommand{\sectionmark}[1]{\markright{\thesection\ #1}{}}
\fancyhead{}
\fancyhead[LE,RO]{\sffamily \thepage}
\fancyhead[RE]{\sffamily \leftmark}
\fancyhead[LO]{\sffamily \rightmark}
\fancyfoot{}%[CE,CO]{\sffamily \thepage}

%\renewcommand{\headrulewidth}{0pt}
% ridefinisco lo stile plain
\fancypagestyle{plain}{ \fancyhead{} \fancyfoot{}
                        \fancyfoot[C]{\sffamily \thepage}
                        \renewcommand{\headrulewidth}{0pt}
                      }

% stile per i titoli
%\allsectionsfont{\sffamily \raggedright}

%definisco stile listati di codice
\lstdefinestyle{mystyle}{
                        backgroundcolor=\color[RGB]{240,240,240}, %Red, Green, Blue from 0 to 255
                        basicstyle=\color{black}\ttfamily\footnotesize,
                        commentstyle=\color{OliveGreen}\textit\tiny,
                        keywordstyle=\color{blue},
                        identifierstyle=\color{black},
                        stringstyle=\color{violet},
                        breakatwhitespace=false,         
                        breaklines=true,
                        captionpos=b, %bottom, top
                        rulecolor=\color{black},
                        frame=trBL, %Top,Right,Bottom,Left (Case sensitive)
                        frameround=tttt, %t=round corner, f=straight corner
                        keepspaces=true,                 
                        numbers=left,                    
                        numbersep=10pt,
                        numberstyle=\color{gray},
                        showspaces=false,                
                        showstringspaces=false,
                        showtabs=false,                  
                        tabsize=1,
                        columns=fullflexible %makes code copy and pastable
                        }

\lstset{style=mystyle}
\lstset{emph={RandomForestClassifier, RandomizedSearchCV, GridSearchCV},emphstyle=\underbar}

\renewcommand{\lstlistingname}{Codice} % Per rinominare la descrizione dei codici
\renewcommand{\lstlistlistingname}{Elenco dei codici} % Per rinominare la descrizione dei codici

% Nuovi comandi
% \input{config/new_commands}

\begin{document}

    \frontmatter

    \myemptypage % Pagina bianca prima del titolo
    
    % Prima pagina ----------------------------------------------------------------------

    \begin{titlepage}

    \begin{center}
        \includegraphics[scale=0.1]{images/titlepage/logo_unipd_small}
        
        \vspace{1cm}
        
        \textsc{\LARGE Universit\`{a} degli Studi di Padova}
        
        \vspace{0.5cm}
        
        \textsc{\large Dipartimento di ...}
        
        \vspace{0.5cm}
        
        \textsc{\large Corso di Laurea ...\\in Ingegneria ...}
        
        \vfill
        
        {\Huge \bfseries Titolo tesi}
        
        \vspace{1cm}
        
        {\LARGE Titolo inglese}
    \end{center}
    
    \vfill
    
    \noindent \textit{\large Relatore:} \\
    \textsc{\large Ch.mo Prof. Nome Cognome}
    %\textsc{\large Chiar.ma Prof.ssa Nome Cognome}
    
    %\vfill
    \vspace{1cm}
    
    \begin{flushright}
        \textit{\large Laureando:}   \\
        \textsc{\large Nome Cognome} \\
        \textsc{1234567}
    \end{flushright}
    
    \vfill
    
    \centering {\large Anno Accademico 20XX/20XX}
    
\end{titlepage}
    
    \thispagestyle{empty} % Pagina bianca dopo il titolo
    \cleardoublepage
    
    % Citazione, ringraziamenti, abstract, indice ---------------------------------------
    
    \mainmatter  
    
    \pagenumbering{roman} % Numerazione romana
    \thispagestyle{empty}
    
    \clearpage{\pagestyle{plain}\cleardoublepage} % Comando per iniziare il capitolo su pagina dispari
    % Citazione
\vspace*{\fill} 
\begin{quote} 
    \centering 
    \begin{flushright}{
                        \slshape  
                        Inserisci qui la tua citazione o dedica.
                      }
    \end{flushright}
\end{quote}
\vspace*{\fill}

\clearpage{\pagestyle{plain}\cleardoublepage}

% Ringraziamenti
\chapter*{Ringraziamenti}
Inserisci gli eventuali ringraziamenti personali.
	
    
    \clearpage{\pagestyle{plain}\cleardoublepage} 
    \input{chapters/abstract}
    
    \clearpage{\pagestyle{plain}\cleardoublepage} 
    \chapter*{Introduzione}
\label{introduzione}

Scrivi qui la tua introduzione.
    
    \clearpage{\pagestyle{plain}\cleardoublepage}
    \tableofcontents % Indice
    
    \clearpage{\pagestyle{plain}\cleardoublepage}
    \phantomsection
    \addcontentsline{toc}{chapter}{Elenco delle figure} % Per inserire nell'indice la lista dei codici (non è automatico)
    \listoffigures % Lista figure
    
    \clearpage{\pagestyle{plain}\cleardoublepage}
    \phantomsection
    \addcontentsline{toc}{chapter}{Elenco delle tabelle} % Per inserire nell'indice la lista dei codici (non è automatico)
    \listoftables % Lista tabelle
    
    \clearpage{\pagestyle{plain}\cleardoublepage}
    \phantomsection
    \addcontentsline{toc}{chapter}{Elenco dei codici} % Per inserire nell'indice la lista dei codici (non è automatico)
    \lstlistoflistings % Lista codici
    
    % Capitoli --------------------------------------------------------------------------
    
    \clearpage{\pagestyle{plain}\cleardoublepage}
    \pagenumbering{arabic} % Numerazione araba per i capitoli
    
    \clearpage{\pagestyle{plain}\cleardoublepage} 
    \chapter{Primo Capitolo} % Nome capitolo
\label{chapter:primo_capitolo} % Label per creare riferimenti al capitolo
La struttura utilizzata in questo template non è obbligatoria, però ritengo che sia molto comoda per evitare di scrivere file troppo lunghi e di avere un controllo migliore sulla struttura. Questa prevede di scrivere l'introduzione al capitolo in un file salvato nella cartella principale e di sviluppare le sezioni all'interno di una cartella. Esempio di richiamo ad un riferimento \cite{Vas}.

\section{Sezione 1}
Ad ogni sezione, in questo template \cite{UsabilityFirst}, corrisponde un file all'interno della cartella relativa al capitolo \cite{Bolognani-Zigliotto}.

\input{chapters/cap1/sezione2.tex} % File in cui verrà scritto il capitolo
    
    \clearpage{\pagestyle{plain}\cleardoublepage}
    \chapter{Immagini e Tabelle} 
\label{chapter:immagini_e_tabelle} 
Metodi più comuni per inserire immagini e esempi di tabelle. \cite{eco:tesi}

\section{Immagine singola}

Per fare riferimento ad un immagine \cite{dirac}, come ad ogni altro elemento a cui viene attribuita una \textit{label} è disponibile il comando \textit{ref}. Riferimento a Figura \ref{fig:figura_test}.
\begin{figure}[!ht]
    \centering
    \includegraphics[width=0.9\textwidth]{images/cap2/test}
    \caption{Didascalia}
    \label{fig:figura_test}
\end{figure}

\section{Immagine multipla}

Inserire più ``sottofigure" in una figura.
\begin{figure}[!ht]
    \centering
    
	%Prima riga
	\begin{subfigure}{0.49\textwidth} %Controllo della posizione in orizzontale
		\includegraphics[width=0.9\textwidth]{images/cap2/test} 
		\caption{}
	\end{subfigure}
	\begin{subfigure}{0.49\textwidth}
		\includegraphics[width=0.9\textwidth]{images/cap2/test}
		\caption{}
	\end{subfigure}
	
	%Seconda riga
	\begin{subfigure}{0.49\textwidth}
		\includegraphics[width=0.9\textwidth]{images/cap2/test} 
		\caption{}
	\end{subfigure}
	\begin{subfigure}{0.49\textwidth}
		\includegraphics[width=0.9\textwidth]{images/cap2/test}
		\caption{}
	\end{subfigure}
    \caption{Esempio di figura composta da 4 figure.}
\end{figure}

%\cleardoublepage

\section{Tabelle}

Nella seguente tabella vengono mostrati alcuni esempi di separazione delle righe. La separazione delle colonne avviene  all'interno delle parentesi graffe dopo il comando \textit{tabular}.
\begin{table}[h]
    \centering
    \begin{tabular}{ ||c|c|c|| } 
        \hline
        cella1  & cella2  & cella3  \\
        \hline
        \hline 
        cella4  & cella5  & cella6  \\ 
        \hline
        cella7  & cella8  & cella9  \\ 
        cella10 & cella11 & cella12 \\ 
        \hline
    \end{tabular}
    \caption{Tabella 1}
    \label{tab:tabella1}
\end{table}

È possibile, inoltre, fissare la dimensione delle colonne \cite{prova2}.
\begin{table}[ht]
    \centering
    \begin{tabular}{ | m{3cm} | m{5cm} | m{1.5cm} | } 
        \hline
        cella1  & cella2  & cella3                      \\
        \hline
        \hline 
        cella4  & cella5  & cella6 cella6 cella6 cella6 \\ 
        \hline
        cella7  & cella8  & cella9                      \\ 
        cella10 & cella11 & cella12                     \\ 
        \hline
    \end{tabular}
    \caption{Tabella 2}
    \label{tab:tabella2}
\end{table} 
    
    \clearpage{\pagestyle{plain}\cleardoublepage}
    \input{chapters/cap3/intro_cap3} 
    
    \clearpage{\pagestyle{plain}\cleardoublepage}
    \chapter{Pseudocodice e codice} 
\label{chapter:codice} 
In questo template per l'inserimento di pseudocodice è stato utilizzato il pacchetto \textit{algpseudocode}. Per quanto riguarda l'inserimento del codice è possibile utilizzare il comando \textit{verb} per inserire in linea oppure \textit{lstlisting} per inserire blocchi di codice \cite{einstein}. 

\input{chapters/cap4/pseudocodice.tex}

\section{Codice}

È possibile inserire codice in linea: \verb!print("Hello World")! \cite{knuthwebsite}.\\
Inoltre è possibile usare l'ambiente \textit{lstlisting} configurando il layout del blocco di codice nel file \textit{layout.tex}. Il codice può essere importato da un file esterno che metteremo nella cartella \textit{code} \cite{mori:tesi}.
\lstinputlisting[   language=Python,
                    caption=Didascalia.,
                    label=ls:codice]
                    {code/randomized_search.py}
            
\lstinputlisting[
                    language=C++,
                    %linerange={0-22},
                    %firstnumber=20,
                    %frame=trL,
                    % frameround=tttt,
                    caption=prova.cpp,
                    label=list:prova
                    ]
                    {code/prova.cpp}
    
    \clearpage{\pagestyle{plain}\cleardoublepage}
    \chapter*{Conclusioni} \addcontentsline{toc}{chapter}{Conclusioni}
\label{conclusioni}

Scrivi qui le tue conclusioni.
    
    % Bibliografia ----------------------------------------------------------------------
    
    \clearpage{\pagestyle{plain}\cleardoublepage}
    
    % This -------------------------------
    %\printbibliography[heading=bibintoc]
    
    % Or this ----------------------------
    % \chapter*{Bibliografia} \addcontentsline{toc}{chapter}{Bibliografia}
    % \printbibliography[heading=subbibintoc, type=book,    title={Libri}]
    % \printbibliography[heading=subbibintoc, type=article, title={Articoli}]
    % \printbibliography[heading=subbibintoc, type=online,  title={Siti}]
    % \printbibliography[heading=subbibintoc, type=misc,    title={Vari}]
    
    % Or this ----------------------------
    \chapter*{Bibliografia} \addcontentsline{toc}{chapter}{Bibliografia} \fancyhead[RE,LO]{\sffamily Bibliografia}
    \printbibliography[nottype=online, heading=subbibliography, keyword={cap1}, title={Capitolo 1}]
    \printbibliography[nottype=online, heading=subbibliography, keyword={cap2}, title={Capitolo 2}]
    \printbibliography[nottype=online, heading=subbibliography, keyword={cap3}, title={Capitolo 3}]
    \printbibliography[nottype=online, heading=subbibliography, keyword={cap4}, title={Capitolo 4}]
    
    \chapter*{Sitografia} \addcontentsline{toc}{chapter}{Sitografia} \fancyhead[RE,LO]{\sffamily Sitografia}
    \printbibliography[type=online, heading=subbibliography, keyword={cap1}, title={Capitolo 1}]
    \printbibliography[type=online, heading=subbibliography, keyword={cap2}, title={Capitolo 2}]
    \printbibliography[type=online, heading=subbibliography, keyword={cap3}, title={Capitolo 3}]
    \printbibliography[type=online, heading=subbibliography, keyword={cap4}, title={Capitolo 4}]
        
\end{document}
