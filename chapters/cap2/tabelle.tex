\section{Tabelle}

Nella seguente tabella vengono mostrati alcuni esempi di separazione delle righe. La separazione delle colonne avviene  all'interno delle parentesi graffe dopo il comando \textit{tabular}.
\begin{table}[h]
    \centering
    \begin{tabular}{ ||c|c|c|| } 
        \hline
        cella1  & cella2  & cella3  \\
        \hline
        \hline 
        cella4  & cella5  & cella6  \\ 
        \hline
        cella7  & cella8  & cella9  \\ 
        cella10 & cella11 & cella12 \\ 
        \hline
    \end{tabular}
    \caption{Tabella 1}
    \label{tab:tabella1}
\end{table}

È possibile, inoltre, fissare la dimensione delle colonne \cite{prova2}.
\begin{table}[ht]
    \centering
    \begin{tabular}{ | m{3cm} | m{5cm} | m{1.5cm} | } 
        \hline
        cella1  & cella2  & cella3                      \\
        \hline
        \hline 
        cella4  & cella5  & cella6 cella6 cella6 cella6 \\ 
        \hline
        cella7  & cella8  & cella9                      \\ 
        cella10 & cella11 & cella12                     \\ 
        \hline
    \end{tabular}
    \caption{Tabella 2}
    \label{tab:tabella2}
\end{table}